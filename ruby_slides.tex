\documentclass{beamer}
\usepackage{beamerthemesplit}
\setbeamercovered{transparent}

\title{Ruby 101}
\author{Dmitry Ratnikov}
\date{\today}

\begin{document}
\frame{\titlepage}

\section[]{Outline}
\frame{\tableofcontents}

\section{Basics}

\begin{frame}
  \frametitle{What is ruby?}
  \begin{block}{Wikipedia says...}
    Ruby is a \alert<2,3>{dynamic}, \alert<2,4>{reflective}, general purpose \alert<2,5>{object-oriented} programming language that combines syntax 
    inspired by \alert<6>{Perl} with \alert<6>{Smalltalk}-like features. 
  \end{block}
\end{frame}
  
\begin{frame}[fragile]
  \frametitle{Variables/Arrays/Hashes}
  \begin{itemize}
    \item<1->{Variables}
      \begin{verbatim}
	foo = "foo"
	size = 5
      \end{verbatim}
    \item<2->{Arrays}
      \begin{verbatim}
	arr = [ 1, 2, 3, 4, 5, 6, 7, 8, 9, 10]
	complex_array = [ "simple_array:", arr ]
      \end{verbatim}
    \item<3->{Hashes}
      \begin{verbatim}
	hash = { :hello => "world" }
	complex_hash = { arr => "Simple array", 
	  :complex_array => { :label => "Complex array", 
	                      :value => complex_array }
      \end{verbatim}
  \end{itemize}
\end{frame}

\begin{frame}[fragile]
  \frametitle{Defining Methods}
  \begin{block}{Making life easier}
    \begin{semiverbatim}
      \uncover<1->{\alert<2>{def} \alert<3>{make\_hash\_key}\alert<2>{(label, value)}}
      \uncover<1->{  \alert<4>{\{ :label => label, :value => value \}} }
      \uncover<1->{\alert<2>{end}}
    \end{semiverbatim}
    \begin{itemize}
      \item<2>Method body is enclosed by \texttt{def} and \texttt{end}. \\ Parameters are a list of variable names.
      \item<3>Method name must be lower case letters and \_. \\ It may be suffixed by ?, = or !.
      \item<4>Last line is returned by the method.
    \end{itemize}
  \end{block}
\end{frame}

\begin{frame}[fragile]
  \frametitle{Control Structures}
  \begin{columns}[t]
    \column{.5\textwidth}
      \begin{block}{if/else}
        \begin{verbatim}
	  if foo
	    puts "foo!"
	  else
	    puts "Not foo..."
	  end

	  puts "Bar" unless bar
	\end{verbatim}
      \end{block}
    \column{.5\textwidth}
      \begin{block}{for/while}
	\begin{verbatim}
	  for i in 0..(arr.size) do
	    puts "#{arr[i]}"
	  end

	  file = File.open("foo.txt")
	  while (line = file.gets)
	    puts "line: #{line.chop}"
	  end
	\end{verbatim}
      \end{block}
  \end{columns}
\end{frame}

\begin{frame}[t,fragile]
  \frametitle{Blocks/iterators}
  \begin{block}{Same thing, but easier}
    \begin{semiverbatim}
      \# Displaying all items in an array
      \uncover<1->{str = ""}
      \uncover<1->{\alert<2>{arr.each} \alert<3>{\{ |item|} \alert<4>{str += "#\{item\} "} \alert<3>{\}}}
      \uncover<1->{puts str.inspect}
    \end{semiverbatim}
  \end{block}
  \begin{onlyenv}<1>
    \begin{block}{Produces:}
      \begin{verbatim}
        "1 2 3 4 5 6 7 8 9 10"
      \end{verbatim}
    \end{block}
  \end{onlyenv}

  \begin{itemize}
    \item<2> Invoke method \texttt{each} on variable \texttt{arr}.
    \item<3> Declare a block that takes one parameter as \texttt{item}.
    \item<4> Specify the body of the block to output the \texttt{item}.
  \end{itemize}
\end{frame}

\begin{frame}[t,fragile]
  \frametitle{Accumulator Style}
  \begin{block}{Even easier}
    \begin{semiverbatim}
      \# Displaying all items in an array
      \uncover<1->{puts \alert<4>{arr.inject("") do \alert<2>{|acc, item|}}}
      \uncover<1->{  \alert<3,4>{"#\{acc\} #\{item\}"}}
      \uncover<1->{\alert<4>{end}}
    \end{semiverbatim}
    \begin{onlyenv}<1>
      \begin{block}{Produces:}
	\begin{verbatim}
	  "1 2 3 4 5 6 7 8 9 10"
	\end{verbatim}
      \end{block}
    \end{onlyenv}
    \begin{onlyenv}<2->
      \begin{itemize}
        \item<2> \texttt{inject}'s block takes two parameters:
        \begin{itemize}
	  \item{acc}: accumulator result from previous call.
	  \item{item}: currently iterated item.
        \end{itemize}
        \item<3> Return of the block is passed in the next acc
        \item<4> Last acc is returned by the \texttt{inject} which is fed into \texttt{puts}.
      \end{itemize}
    \end{onlyenv}
    \end{block}
\end{frame}
  

\section{Advanced Basics}
\begin{frame}
\end{frame}

\section{Duck Typing}
\begin{frame}
\end{frame}

\section{Meta Stuff}
\begin{frame}
\end{frame}

\end{document}
