\documentclass{beamer}
\usepackage{beamerthemesplit}
\setbeamercovered{transparent}

\title{Ruby 101}
\author{Dmitry Ratnikov}
\date{\today}

\begin{document}
\frame{\titlepage}

\section[]{Outline}
\frame{\tableofcontents}

\section{Basics}

\begin{frame}
  \frametitle{What is ruby?}
  \begin{block}{Wikipedia says...}
    Ruby is a \alert<2,3>{dynamic}, \alert<2,4>{reflective}, general purpose \alert<2,5>{object-oriented} programming language that combines syntax 
    inspired by \alert<6>{Perl} with \alert<6>{Smalltalk}-like features. 
  \end{block}
\end{frame}
  
\begin{frame}[fragile]
  \frametitle{Variables/Arrays/Hashes}
  \begin{itemize}
    \item<1->{Variables}
      \begin{verbatim}
	pickaxe_book = "Programming Ruby"
	cs_bible = "Art of Computer Programming"
	js_book = "Javascript: The Good Parts"
      \end{verbatim}
    \item<2->{Arrays}
      \begin{verbatim}
	available_books = [ pickaxe_book, cs_bible ]
      \end{verbatim}
    \item<3->{Hashes}
      \begin{verbatim}
	library = {
	  :available => available_books,
	  :checked_out => [ js_book ] 
	}
      \end{verbatim}
  \end{itemize}
\end{frame}

\begin{frame}[fragile]
  \frametitle{Defining Methods}
  \begin{block}{Making life easier}
    \begin{semiverbatim}
      \uncover<1->{\alert<2>{def} \alert<3>{available?}\alert<2>{(library, book\_name)}}
      \uncover<1->{  \alert<4>{library[:available].include?(book\_name)} }
      \uncover<1->{\alert<2>{end}}
    \end{semiverbatim}
    \begin{itemize}
      \item<2>Method body is enclosed by \texttt{def} and \texttt{end}. \\ Parameters are a list of variable names.
      \item<3>Method name must be lower case letters and \_. \\ It may be suffixed by ?, = or !.
      \item<4>Last line is returned by the method.
    \end{itemize}
  \end{block}
\end{frame}

\begin{frame}[fragile]
  \frametitle{Conditionals}
  \begin{block}{if/else syntax}
    \begin{verbatim}
      if available?(library, "Art of War")
        puts "Sun Tzu's Art of War is available."
      else
        puts "Art of War is not available."
        puts "Try later..."
      end
    \end{verbatim}
    \begin{block}{Can be inlined:}
      \verb|      puts "yay" unless boo?|
    \end{block}
  \end{block}
\end{frame}

\begin{frame}[fragile]
  \frametitle{Loops}

  \begin{block}{For/While loops:}
    \begin{verbatim}
      file = File.open("checked_out_backup.txt")
      while (book = file.gets)
        library[:checked_out] << book
      end
    \end{verbatim}
    \begin{verbatim}
      str = ""
      for i in 0..(library[:checked_out].size) do
        str += "#{library[:checked_out][i]} "
      end
      puts "Checked out books: #{str}"
    \end{verbatim}
  \end{block}
\end{frame}

\begin{frame}[t,fragile]
  \frametitle{Blocks/iterators}
  \begin{block}{Same as for, but easier}
    \begin{semiverbatim}
      \uncover<1->{str = ""}
      \uncover<1->{\alert<2>{library[:checked\_out].each} \alert<3>{do |book|}} 
      \uncover<1->{  \alert<4>{str += "#\{book\} "}}
      \uncover<1->{\alert<3>{end}}
      \uncover<1->{puts "Checked out books: #\{str\}}
    \end{semiverbatim}
  \end{block}

  \begin{itemize}
    \item<2> Invoke method \texttt{each} on the checked out books array.
    \item<3> Declare a block that takes one parameter as \texttt{book}.
    \item<4> Specify the body of the block to append the \texttt{book} to the \texttt{str}.
  \end{itemize}
\end{frame}

\begin{frame}[t,fragile]
  \frametitle{Accumulator Style}
  \begin{block}{Even easier}
    \begin{semiverbatim}
      \uncover<1->{str = \alert<4>{arr.inject("") do \alert<2>{|acc, item|}}}
      \uncover<1->{  \alert<3,4>{"#\{acc\} #\{item\}"}}
      \uncover<1->{\alert<4>{end}}
      \uncover<1->{puts "Checked out books: #\{str\}"}
    \end{semiverbatim}
    \begin{itemize}
      \item<2> \texttt{inject}'s block takes accumulator and book parameters.
      \item<3> Return of the block is passed in the next acc
      \item<4> Last acc is returned by the \texttt{inject} which is assigned to \texttt{str}.
    \end{itemize}
  \end{block}
\end{frame}

\begin{frame}[t,fragile]
  \frametitle{Declaring a Method With Block}
  \begin{block}{Sample implementation of \texttt{inject}}
    \begin{semiverbatim}
      \uncover<1->{def available\_inject(library, init, \alert<2>{\&block})}
      \uncover<1->{  \alert<2>{raise "Block missing" unless block\_given?}}
      \uncover<1->{  arr = library[:checked\_out]}
      \uncover<1->{  acc = init}
      \uncover<1->{  arr.each \{ |item| acc = \alert<3>{yield(acc, item)} \} }
      \uncover<1->{  acc}
      \uncover<1->{end}
    \end{semiverbatim}
  \end{block}

  \begin{onlyenv}<2->
    \begin{itemize}
      \only<1>{\item{}}
      \begin{onlyenv}<2>
	\item Block is passed using \&
	\item \texttt{block\_given?} returns whether method was provided a block.
      \end{onlyenv}
      \only<3>{\item \texttt{yield} yields control to the provided block with specified parameters.}
    \end{itemize}
  \end{onlyenv}
\end{frame}

\section{Object orientation}
\begin{frame}
  \frametitle{Declaring a Class}
  \begin{block}{Syntax:}
  \begin{itemize}
    \item Classes are declared by keyword \texttt{class}
    \item Instance variables are specified by prepending '@' to \\ a variable name (e.g. \texttt{@foo})
    \item Class variables are specified by prepending '@@' (e.g. \texttt{@@bar})
  \end{itemize}
  \end{block}
\end{frame}

\begin{frame}[t,fragile]
  \frametitle{Declaring a Class (cont.)}
  \begin{example}
    \begin{semiverbatim}
      \uncover<1->{class \alert<2>{Circle < GeoFigure}}
      \uncover<1->{  @@name = 'Circle'}
      \uncover<1->{  def radius; @radius end}
      \uncover<1->{  def radius=(new\_radius)}
      \uncover<1->{    @radius = new\_radius; @radius}
      \uncover<1->{  end}
      \uncover<1->{end}
    \end{semiverbatim}
  \end{example}

  \begin{itemize}
    \item<2> Declares that \texttt{Circle} inherits from \texttt{GeoFigure}
  \end{itemize}
\end{frame}

\begin{frame}
  \frametitle{Modules}
  \begin{definition}[Module]
    Module is an encapsulated piece of functionality.
  \end{definition}
\end{frame}

\section{Duck Typing}
\begin{frame}
\end{frame}

\section{Meta Stuff}
\begin{frame}
\end{frame}

\end{document}
