\documentclass[10]{beamer}
\usepackage{beamerthemesplit}
\setbeamercovered{transparent}

\newcommand<>{\marked}[1]{{\color#2{blue}#1}}
\newcommand<>{\changed}[1]{{\color#2[rgb]{0,.4,0}#1}}

\title{Ruby}
\author{Dmitry Ratnikov}
\date{\today}

\begin{document}
\frame{\titlepage}

\section[]{}

\begin{frame}
  \frametitle{Outline}
  \begin{block}{General plan for today:}
    \begin{itemize}
      \visible<2->{\item Introduce you to the ruby world.}
      \visible<4->{\item Show how one can do some cool things with it.}
      \visible<5->{\item Hopefully you'll get all excited about it.}
    \end{itemize}
  \end{block}
\end{frame}

\section{Ruby intro}

\begin{frame}
  \frametitle{What is ruby?}
  \begin{block}{Wikipedia says...}
    \begin{visibleenv}<2->
    Ruby is a \alert<3,4>{dynamic}, \alert<3,5>{reflective}, general purpose \alert<3,6>{object-oriented} programming language that combines syntax 
    inspired by \alert<3,7>{Perl} with \alert<3,7>{Smalltalk}-like features. 
    \end{visibleenv}
  \end{block}
\end{frame}

\subsection{Imperative basics}

\begin{frame}[fragile]
  \frametitle{In ruby you can...}
  \begin{itemize}
    \item<2->{Create variables:}
      \begin{verbatim}
  pickaxe_book = "Programming Ruby"
  cs_bible = "Art of Computer Programming"
  js_book = "Javascript: The Good Parts"
    \end{verbatim}
    \item<3->{Create arrays:}
    \begin{verbatim}
  available_books = [ pickaxe_book, cs_bible ]
    \end{verbatim}
    \item<4->{Create hashes:}
    \begin{verbatim}
  library = {
    :available => available_books,
    :checked_out => [ js_book ] 
  }
      \end{verbatim}
  \end{itemize}
\end{frame}

\begin{frame}[fragile]
  \frametitle{In ruby you can...}
  \begin{itemize}
    \item<1->{Create methods:}
    \begin{verbatim}
  def available?(library, book_name)
    library[:available].include?(book_name)
  end
    \end{verbatim}
    \item<2->{Invoke methods (and print to screen):}
  \begin{verbatim}
  str = "Available: #{available? library, book_name}"
  puts str
    \end{verbatim}
  \end{itemize}
\end{frame}

\begin{frame}[fragile]
  \frametitle{In ruby you can...}
  \begin{itemize}
    \item<1->{Do if/else:}
    \begin{verbatim}
  if available?(library, "Art of War")
    puts "Sun Tzu's Art of War is available."
  else
    puts "Art of War is not available."
    puts "Try later..."
  end
    \end{verbatim}
    \item<2->{Inline conditionals:}
    \begin{verbatim}
  puts "Hello world" unless lhc_started?
    \end{verbatim}
  \end{itemize}
\end{frame}

\begin{frame}[fragile]
  \frametitle{In ruby you can...}
  \begin{itemize}
    \item<1->{Do while loops:}
    \begin{verbatim}
  file = File.open("checked_out_backup.txt")
  while (book = file.gets)
    library[:checked_out] << book
  end
    \end{verbatim}

    \item<2->{Do for loops:}
    \begin{verbatim}
  str = ""
  for i in 0..(library[:checked_out].size) do
    str += "#{library[:checked_out][i]} "
  end
  puts "Checked out books: #{str}"
    \end{verbatim}
  \end{itemize}
\end{frame}

\subsection{Blocks and procs}

\begin{frame}[t,fragile]
  \frametitle{Using blocks}
  \begin{block}{But ruby allows simpler iteration via blocks:}
    \begin{semiverbatim}
  \uncover<1->{str = ""}
  \uncover<1->{\alert<3>{books.each} \alert<2>{do |book|}} 
  \uncover<1->{  \alert<2>{str += "#\{book\} "}}
  \uncover<1->{\alert<2>{end}}
  \uncover<1->{puts "Checked out books: #\{str\}}
    \end{semiverbatim}
  \end{block}

  \begin{itemize}
    \item<2-> Define what to do with an element of the array in a block.
    \item<3-> Apply that block to each element of the array.
  \end{itemize}
\end{frame}

\begin{frame}[t,fragile]
  \frametitle{Using accumulating blocks}
  \begin{block}{Might as well use accumulator style:}
    \begin{semiverbatim}
  \uncover<1->{str = \alert<4>{arr.inject("") do \alert<2>{|acc, item|}}}
  \uncover<1->{  \alert<3,4>{"#\{acc\} #\{item\}"}}
  \uncover<1->{\alert<4>{end}}
  \uncover<1->{puts "Checked out books: #\{str\}"}
    \end{semiverbatim}
    \begin{itemize}
      \item<2-> \texttt{inject}'s block takes accumulator and book parameters.
      \item<3-> Return of the block is passed in the next acc
      \item<4-> Last acc is returned by the \texttt{inject} which is assigned to \texttt{str}.
    \end{itemize}
  \end{block}
\end{frame}

\begin{frame}[t,fragile]
  \frametitle{Declaring a Method With Block}
  \begin{block}{Sample implementation of \texttt{inject}}
    \begin{semiverbatim}
      \uncover<1->{def available\_inject(library, init, \alert<2>{\&block})}
      \uncover<1->{  \alert<2>{raise "Block missing" unless block\_given?}}
      \uncover<1->{  arr = library[:checked\_out]}
      \uncover<1->{  acc = init}
      \uncover<1->{  arr.each \{ |item| acc = \alert<3>{yield(acc, item)} \} }
      \uncover<1->{  acc}
      \uncover<1->{end}
    \end{semiverbatim}
  \end{block}

  \begin{onlyenv}<2->
    \begin{itemize}
      \only<1>{\item{}}
      \begin{onlyenv}<2>
	\item Block is passed using \&
	\item \texttt{block\_given?} returns whether method was provided a block.
      \end{onlyenv}
      \only<3>{\item \texttt{yield} yields control to the provided block with specified parameters.}
    \end{itemize}
  \end{onlyenv}
\end{frame}

\subsection{Object-oriented basics}

\begin{frame}[fragile]
  \frametitle{In ruby you can...}
  \begin{itemize}
    \item<1->{Create classes:}
    \begin{verbatim}
  class Foo
    @@description = "The most important class"
    def foo
      @foo ||= busily_lookup_foo
      @foo
    end
  end
    \end{verbatim}
  \end{itemize}
\end{frame}

\begin{frame}[fragile]
  \frametitle{In ruby you can...}
  \begin{itemize}
    \item<1->{Extend classes:}
    \begin{semiverbatim}
  class Bar < Foo
    \alert<2>{attr\_accessor :bar}
    \alert<3>{def initialize options = \{\}}
      self.bar = options.delete(:bar)
    \alert<3>{end}
  end
    \end{semiverbatim}
    \begin{visibleenv}<2->
    \begin{itemize}
      \item<2-> Defines a reader/writer for bar attribute.
      \item<3-> Defines constructor for \texttt{Bar} class.
    \end{itemize}
    \end{visibleenv}
  \end{itemize}
\end{frame}

\section{Library project}

\subsection{Specification}

\begin{frame}
  \frametitle{Library manager}
  \begin{block}{Boss says:}
    \begin{quote}
    Make a library managing application.
    \end{quote}
  \end{block}
  \begin{visibleenv}<2->
  \begin{block}{After researching the topic, we deduce that the application should:}
    \begin{itemize}
      \item<2-> Track books and games for check in and check out. 
      \item<3-> Provide cataloging of available items.
    \end{itemize}
  \end{block}
  \end{visibleenv}
\end{frame}

\subsection{Book class}

\begin{frame}[fragile]
  \frametitle{app/models/book.rb}
  \begin{verbatim}
  class Book
    attr_accessor :name, :author, :isbn
    def description
      "#{name} by #{author}"
    end

    def initialize options = {}
      self.name = options.delete(:name) or
        raise("Need name.")
      self.author = options.delete(:author) or 
        raise("Need author.")
      self.isbn = options.delete(:isbn)
    end
  end
  \end{verbatim}
\end{frame}

\begin{frame}[fragile]
  \frametitle{test/unit/book\_test.rb}
  \begin{verbatim}
  class BookTest < Test::Unit::TestCase
    def test_description
      assert_equal "foo by bar", 
        Book.new(:name => "foo", :author => "bar")
    end
    def test_required_options
      asser_raise(RuntimeError, "should need name) do
        Book.new :author => "foo"
      end
      assert_raise(RuntimeError, "should need author) do
        Book.new :name => "bar"
      end
    end
  end
  \end{verbatim}
\end{frame}

\subsection{Game class}

\begin{frame}[fragile]
  \frametitle{app/models/game.rb}
    \begin{semiverbatim}
  class Game
    attr\_accessor :name, :author, :platform
    \alert<2>{def description
      "#\{name\} by #\{author\}"
    end}

    def initialize options = \{\}
      \alert<2>{self.name = options.delete(:name) or
        raise("Need name.")
      self.author = options.delete(:author) or 
        raise("Need author.")}
      self.platform = options.delete(:platform)
    end
    \end{semiverbatim}
\end{frame}

\begin{frame}[fragile]
  \frametitle{app/models/game.rb 2.0}
  \begin{semiverbatim}
  class Game \alert<2>{< Book}
    \alert<3>{attr\_accessor :platform}
    def initialize options = \{\}
      super(options)
      self.platform = options.delete(:platform)
    end
  end
  \end{semiverbatim}
  \begin{itemize}
    \item<2-> Now Game inherits all methods from Book class.
    \item<3-> Has an additional platform attribute.
    \item<4-> But... inherits isbn, which games do not really have.
  \end{itemize}
\end{frame}

\begin{frame}
  \frametitle{How do we fix it?}
  \begin{block}{What we really want is:}
    \begin{itemize}
      \item<1-> Encapsulate the `Catalogable' functionality that:
	\begin{itemize}
	  \item<2-> does book-keeping of having a name
	  \item<3-> does book-keeping of having an author.
	  \item<4-> builds description out of the name and author.
	\end{itemize}
      \item<5-> Include that functionality in the \texttt{Book} and \texttt{Game} classes.
    \end{itemize}
  \end{block}
  \begin{visibleenv}<6>
    \begin{block}{Solution:}
      Modules
    \end{block}
  \end{visibleenv}
\end{frame}

\subsection{Modules}

\begin{frame}
  \frametitle{Planning the module}
  \begin{definition}[Module]
    A Module is a collection of methods and constants.
  \end{definition}

  \begin{block}{Game plan:}<2->
    \begin{itemize}
      \item<2-> Create \texttt{Catalogable} module that provides desired functionality.
      \item<3-> Include it into \texttt{Book} and \texttt{Game} classes.
    \end{itemize}
  \end{block}
\end{frame}

\begin{frame}[fragile]
  \frametitle{app/models/catalogable.rb}
  \begin{verbatim}
  module Catalogable
    attr_accessor :name, :author
    def description
      "#{name} by #{author}"
    end
  end
  \end{verbatim}
\end{frame}

\begin{frame}[fragile]
  \frametitle{Using a module}
  \begin{semiverbatim}
    \uncover<1->{class Book}
    \uncover<1->{  include Catalogable}
    \uncover<1->{  attr\_accessor :isbn}
    \uncover<1->{end}
    \uncover<1->{class Game}
    \uncover<1->{  include Catalogable}
    \uncover<1->{  attr\_accessor :platform}
    \uncover<1->{end}
  \end{semiverbatim}

  \begin{itemize}
    \item \texttt{Book} and \texttt{Game} are now independent and shared functionality is abstracted neatly in the \texttt{Catalogable} module.
  \end{itemize}
\end{frame}

\section{The cool bits}

\begin{frame}
  \begin{block}{Ruby: The cool bits}
    \begin{itemize}
      \item Some ruby principles
      \item Extending ruby
      \item DRYing things up with dynamic method generation.
    \end{itemize}
  \end{block}
\end{frame}

\subsection{Ruby principles}

\begin{frame}
  \frametitle{Principle \#1 (DRY)}
  \begin{block}{``Don't Repeat Yourself'' principle:}
    If you have to do something more than once, abstract it away.
  \end{block}
  \begin{visibleenv}<2->
  \begin{block}{Why?}
      \begin{itemize}
      \visible<3->{\item Code duplication means you wrote it at least twice.}
      \visible<4->{\item Code duplication reduces clarity.}
      \visible<5->{\item Code duplication is much harder to keep in sync.}
    \end{itemize}
  \end{block}
  \end{visibleenv}
\end{frame}

\begin{frame}
  \frametitle{Principle \#2 (YAGNI)}
  \begin{block}{``You Ain't Gonna Need It'' principle:}
    Always implement things when you actually need them, \\
    never when you just foresee that you need them.
  \end{block}
  \begin{visibleenv}<2->
  \begin{block}{Why?}
    \begin{itemize}
    \visible<3->{\item Time is better spent on something you actually need}
    \visible<4->{\item What you predict will happen usually is not what really happens.}
    \visible<5->{\item By the time you will need it, you will know the problem better.} 
    \end{itemize}
  \end{block}
  \end{visibleenv}
  \begin{visibleenv}
  \begin{block}{But what about DRY?}
    \begin{itemize}
      \item DRY things up when you actually repeat yourself, \\
        not when you think you may repeat yourself.
    \end{itemize}
  \end{block}
  \end{visibleenv}
\end{frame}

\begin{frame}
  \frametitle{Principle \#3 (Duck typing)}
  \begin{block}{Duck typing principle:}
    If it walks like a duck and quacks like a duck, it is a duck.
  \end{block}
  \begin{visibleenv}<2->
  \begin{block}{In practice that means}
    \begin{itemize}
      \visible<2->{\item What's important is what an object does, not what it is.}
      \visible<3->{\item In duck-typed languages, interfaces are implicitly specified by defined methods.}
    \end{itemize}
  \end{block}
  \end{visibleenv}
\end{frame}

\begin{frame}[fragile]
  \frametitle{Duck typing in use}
  \begin{semiverbatim}
    \uncover<1->{class Library}
    \uncover<1->{  attr\_accessor :items}
    \uncover<1->{  def catalog}
    \uncover<1->{    items.\alert<2>{map} \{ |i| \alert<3>{i.description} \}.\alert<4>{join ", "}}
    \uncover<1->{  end}
    \uncover<1->{end}
  \end{semiverbatim}
  \begin{visibleenv}<2->
  \begin{block}{Only things this \texttt{Library} implementation cares about:}
    \begin{itemize}
      \item<3-> \texttt{items} responds to \texttt{map}.
      \item<4-> Each element of \texttt{items} responds to \texttt{description}.
      \item<5-> Whatever \texttt{i.description} returns must be concatenatable by \texttt{join}.
    \end{itemize}
  \end{block}
  \end{visibleenv}
\end{frame}

\subsection{Extending Ruby}

\begin{frame}
  \frametitle{Extending ruby}
  \begin{block}{Question:}
    What if I want a method that ruby doesn't have?
  \end{block}
  \begin{visibleenv}<2->
  \begin{block}{Answer:}
    Extend ruby to have it!
  \end{block}
  \end{visibleenv}
\end{frame}

\begin{frame}[fragile]
  \frametitle{Adding \texttt{Fixnum\#inject}}
  \begin{block}{Suppose we want to be able to do:}
  \begin{verbatim}
     sorted_profiles = 50.inject([]) do |acc| 
       acc + [Profile.random!]
     end.sort_by { |p| p.name }
  \end{verbatim}
  \end{block}

  \begin{block}{But...}<2>
    \begin{itemize}
      \item But ruby doesn't have Fixnum\#inject
    \end{itemize}
  \end{block}
\end{frame}

\begin{frame}[fragile]
  \frametitle{lib/extensions/fixnum.rb}
  \begin{block}{No problem:}
  \begin{verbatim}
    class Fixnum
      def inject(init = nil, &block)
        raise "Block missing" unless block_given?
        acc = init
        for i in 0..(self-1) do
          init = yield(init, i)
        end
      end
    end
  \end{verbatim}
  \end{block}
\end{frame}

\subsection{Dynamic ruby}

\begin{frame}[t,fragile]
  \frametitle{Case study}
    \begin{verbatim}
module JavascriptHelper
  def author_js author
    "var author = constructAuthor(#{author.name})"
  end
  def book_js book
    "var book = constructBook(#{book.name})"
  end  
  def author_js_tag author
    script_tag author_js(author) 
  end
  def book_js_tag book
    script_tag book_js(book) 
  end
end
  \end{verbatim}
\end{frame}

\begin{frame}[t,fragile]
  \frametitle{app/helpers/javascript\_helper.rb}
  \begin{onlyenv}<1-3>
  \begin{semiverbatim}
  \# ...
  \alert<2,3>{def} author\_js\alert<2,3>{\_tag} \marked<3>{author}
    \alert<2,3>{script\_tag} author\_js\alert<2,3>{(}\marked<3>{author}\alert<2,3>{)}
  \alert<2,3>{end}
  \alert<2,3>{def} book\_js\alert<2,3>{\_tag} \marked<3>{book}
    \alert<2,3>{script\_tag} book\_js\alert<2,3>{(}\marked<3>{book}\alert<2,3>{)}
  \alert<2,3>{end}
  \end{semiverbatim}
  \begin{itemize}
    \item<2-> Both methods have a very similar structure.
    \item<3-> Both methods do the same things with their arguments.
  \end{itemize}
  \end{onlyenv}
  \begin{onlyenv}<4-5>
  \begin{block}{After refactoring:}
  \begin{semiverbatim}
  [:book\_js, :author\_js].each do |js\_method|
    define\_method "\#\{js\_method\}\_tag" do |item|
      script\_tag send(js\_method, item)
    end
  end
  \end{semiverbatim}
  \begin{itemize}
    \item<5> What about making all methods that end in \texttt{\_js} to \\ have a \texttt{\_tag} counterpart?
  \end{itemize}
  \end{block}
  \end{onlyenv}
  \begin{onlyenv}<6->
  \begin{block}{After second refactoring:}
  \begin{semiverbatim}
  instance\_methods.select do |m| 
    m =~ /\_js\$/ 
  end\note{Have to space break due to slide width}.each do |js\_method|
    define\_method "#\{js\_method\}\_tag" do |*args|
      script\_tag send(js\_method, *args)
    end
  end
  \end{semiverbatim}
  \end{block}
  \end{onlyenv}
\end{frame}

\section[]{}

\begin{frame}
  \frametitle{More cool ruby things}
  \begin{block}{Other things one can do with ruby:}
    \begin{itemize}
      \item Fake things when testing things.
      \item Generate classes and modules dynamically.
      \item Use hooks in ruby to provide aspect-oriented behavior.
    \end{itemize}
  \end{block}
\end{frame}

\begin{frame}
  \frametitle{Resources}
    \begin{itemize}
      \item This presentation git clone url: \url{git://github.com/ratnikov/ruby-presentation.git}
      \item Ruby documentation: \url{http://ruby-doc.org}
      \item \#ruby-lang on IRC 
    \end{itemize}
\end{frame}

\begin{frame}
  \frametitle{Questions?}
  \begin{centering}
    \usebeamercolor{titlelike}
    \textbf{\insertframetitle}
  \end{centering}
\end{frame}

\end{document}
