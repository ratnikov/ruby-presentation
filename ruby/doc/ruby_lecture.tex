\documentclass{article}

\title{Ruby lecture layout}
\author{Dmitry Ratnikov}
\date{\today}

\begin{document}
\maketitle

\section{Abstract}

Ruby is a dynamic, reflective, general purpose object-oriented programming language that allows fast and DRY development of large and complex applications. While being
fully object-oriented language, Ruby also supports higher order functions appealing to the functional language crowd and allowing to get the best of the both worlds.

In this talk I will go over the ruby basics and then dive in the cool bits of the ruby world: ducktyping and meta-programming.
\section{Ruby Basics}

\subsection{Variables/Arrays/Hashes}
\subsection{Control structures}
\begin{itemize}
  \item if/else/unless
  \item for/while
  \item case
\end{itemize}

\subsection{Regular Expressions (nod to Perl folks)}

\subsection{Blocks/procs/iteration}

\begin{itemize}
  \item Short/long hand notation
\end{itemize}

\subsection{puts/gets}

\section{Advanced Basics}

\subsection{Classes}
\begin{itemize}
  \item methods are messages (not to Smalltalk folks)
  \item instance variables
  \item class methods/variables
  \item access control (yes, it's different than Java)
\end{itemize}

\subsection{Functional style}
\begin{itemize}
  \item procs revisited
  \item accumulator based iteration
  \item map
\end{itemize}

\section{Duck typing}

\subsection{What's duck typing}
  If it walks like a duck and talks like a duck, it is a duck.

\subsection{Interfaces in ruby}
  \begin{itemize}
    \item Things about interfaces they don't tell you in Java 101
    \item Implicit interfaces in ruby
  \end{itemize}

\subsection{Dynamic class overriding}
  \begin{itemize}
    \item to\_foo protocol
    \item dangers of overriding (If prog isn't loaded fully)
  \end{itemize}

\section{Dynamic stuff}

\subsection{metaclass vs. class \& self}
\begin{itemize}
  \item How methods are found
  \item method\_missing
\end{itemize}

\subsection{Dynamic creation of methods}
\begin{itemize}
  \item define\_method vs. def
\end{itemize}
\subsection{Dynamic creation of modules}
\begin{itemize}
  \item Module.new
  \item How to mix in state (closures)
\end{itemize}

\end{document}
